\chapter{Guten Tag}%
\label{cha:guten_tag}

\noindent Guten~Tag adds a \emph{Tags} palette to the top right corner of your Glyphs window.
There, you can edit the tags for the currently selected glyphs.

\bigbreak\noindent%
\Screenshot[window]{Images/Glyphs window with empty tags field.png}
\bigbreak

\section{Adding \& Removing Tags}%
\label{sec:adding_and_removing_tags}

\Screenshot*[1656, 1086, 112, 232][1.5]{Images/Type tag name.png}\hfill
\Screenshot*[1656, 1086, 112, 232][1.5]{Images/Tokenize tag name.png}\hfill
\Screenshot*[1656, 1086, 112, 232][1.5]{Images/Type tag name with suggestion.png}

\medbreak\noindent Click on the tags field and start typing the name of a tag.
Tags can include letters, numbers, spaces, and other punctuation marks.
Only the comma (\texttt{,}) is special.
Typing a comma or pressing Return will add the tag and display its name in a blue token.
An autocompletion menu pops up when typing a tag that is already used within the font.
Accept a suggestion by pressing Return.

\noindent%
\Screenshot*[1656, 1086, 112, 232][1.5]{Images/Tag name selected.png}\hfill
\Screenshot*[1656, 1086, 112, 232][1.5]{Images/Tokenize tag name.png}\hfill
\Screenshot*[1656, 1086, 112, 232][1.5]{Images/Glyphs window with empty tags field.png}\hfill

\medbreak\noindent You can edit tags like normal text: drag the mouse cursor to select tags or use keyboard commands like \KeyboardEquivalent[shift]{←} or \KeyboardEquivalent[shift]{→}.
Selected tags appear with a white font on a dark blue background.
Delete tags by pressing the Delete~\KeyboardEquivalent{delete left} key.

\section{Batch Editing Tags}%
\label{sec:batch_editing_tags}

Editing tags from the tags field is possible when you have only one glyph selected or all selected glyphs have the same tags.
If, however, the tags of the selected glyphs are different then Guten~Tag can no longer show a unified tags field.
Instead, \enquote{Multiple Values} is displayed in the tags field and an additional row of controls appears at the bottom of the Guten~Tag palette.

\bigbreak\noindent%
\Screenshot[window]{Images/Multiple Selection.png}

\bigbreak\noindent Typing into a multiple-values field will overwrite the tags for all selected glyphs.
If you want to change tags individually without modifying any of the other tags, use the three buttons below the tags field.

\subsection{Add Tags}%
\label{sub:add_tags}

Pressing the Plus \ButtonSymbol{Plus} button allows you to add tags to the currently selected glyphs.
All existing tags of the selected glyphs will be maintained.

\medbreak\noindent%
\Screenshot*[286, 550, 286, 500]{Images/Batch Add Tags.png}

\subsection{Remove Tags}%
\label{sub:remove_tags}

Removing tags works similarly.
Press the Minus \ButtonSymbol{Minus} button and enter the tags that you want to remove from all selected glyphs.

\medbreak\noindent%
\Screenshot*[286, 550, 286, 500]{Images/Batch Remove Tags.png}

\subsection{Rename Tags}%
\label{sub:rename_tags}

You can also batch rename tags by pressing \Label{Rename}.
Select the tag that you want to rename and enter its new name.
You can also rename a tag to the name of another tag\kern0.05em, merging the two.

\medbreak\noindent%
\Screenshot*[286, 550, 286, 500]{Images/Batch Rename Tags.png}
