\documentclass[%
  DIV = 9,%
  twoside = false,%
  fontsize = 11.5,%
  numbers = noenddot]{scrbook}
\usepackage{luacode}
\usepackage{fontspec}
\usepackage{polyglossia}
\setdefaultlanguage{english}
\usepackage[%
  babel = true,%
  letterspace = 38]{microtype}
\usepackage{xcolor}
\usepackage{amsmath}
\usepackage{csquotes}
\usepackage{graphicx}
\usepackage{pdfpages}
\usepackage{wrapfig}
\usepackage[%
  pages = some]{background}
\usepackage{listings}
\usepackage[%
  unicode,%
  hidelinks]{hyperref}
\usepackage{titlesec}
\usepackage{fnpct}
\usepackage{xparse}
\usepackage{metalogo}

\LuaCodeDebugOn

\begin{luacode*}
  function dirname(path)
    if path:match(".-/.-") then
      local path = path:gsub("(.*/)(.*)", "%1")
      return path
    end
    
    return ""
  end
  
  function run_command(cmd)
    local f = io.popen(cmd, "r")
    local str = f:read("*a")
    f:close()
    return str
  end
  
  function trim_whitespace(s)
     return (s:gsub("^%s*(.-)%s*$", "%1"))
  end
  
  function as_points(dimention)
    return dimention / 65536
  end
  
  function font_root(font_name)
    local font_path = run_command("fontpath \""..font_name.."\"")
    return dirname(font_path)
  end
  
  function extra_font_root()
    local path = run_command("extrafontpath")
    root = path:gsub("%s*$", "")
    return root.."/"
  end
  
  function macos_userinterface_path(pathcode)
    local elements = {}
    
    for element in string.gmatch(pathcode, "([^>]+)") do
      element_label = trim_whitespace(element)
      table.insert(elements, "\\PathElement{"..element_label.."}")
    end
    
    tex.print(table.concat(elements, "\\PathSeparator"))
  end
  
  function parse_crop_code(code)
    crop = {}
    
    for offset in string.gmatch(code, "%d+") do
      table.insert(crop, offset)
    end
    
    return crop[1], crop[2], crop[3], crop[4]
  end
  
  function screenshot_crop_for_frame(frame)
    if frame == "window" then
      return 111, 147, 111, 75
    elseif frame == "panel" then
      return 46, 54, 46, 38
    elseif frame == "none" then
      return 0, 0, 0, 0
    else
      return parse_crop_code(frame)
    end
  end
  
  function screenshot_graphic(file, is_crop, scale, frame, top_compensation, bottom_compensation)
    local min_x, min_y, inverse_max_x, inverse_max_y = screenshot_crop_for_frame(frame)
    
    if top_compensation > 0 then
      inverse_max_y = inverse_max_y / top_compensation
    end
    if bottom_compensation > 0 then
      min_y = min_y / bottom_compensation
    end
    
    local image = img.scan { filename = file }
    local res_x = image.xres
    local res_y = image.yres
    local size_x = image.xsize
    local size_y = image.ysize
    
    local unit = image.width / size_x
    local frame_size = image.width - (min_x + inverse_max_x) * unit
    
    local image_width = frame_size / scale
    local textwidth = tex.getdimen("textwidth")
    local width = math.min(image_width, textwidth)
    
    local max_x = size_x - inverse_max_x
    local max_y = size_y - inverse_max_y
    
    width = as_points(width)
    min_x = min_x / res_x
    min_y = min_y / res_y
    max_x = max_x / res_x
    max_y = max_y / res_y
    
    local options = ""
    
    if is_crop then
      options = options..", clip = true"
    end
    
    return "\\includegraphics[width = "..width.."pt, viewport = "..min_x.."in "..min_y.."in "..max_x.."in "..max_y.."in"..options.."]{"..file.."}"
  end
  
  function modifier_key_symbols(code)
    local is_shift = false
    local is_control = false
    local is_option = false
    local is_command = false
    
    for element in string.gmatch(code, "(%S+)") do
      if element == "shift" then
        is_shift = true
      elseif element == "control" then
        is_control = true
      elseif element == "option" then
        is_option = true
      elseif element == "command" then
        is_command = true
      end
    end
    
    symbols = ""
    
    if is_shift then
      symbols = symbols.."⇧"
    elseif is_control then
      symbols = symbols.."⌃"
    elseif is_option then
      symbols = symbols.."⌥"
    elseif is_command then
      symbols = symbols.."⌘"
    end
    
    return symbols
  end
  
  function key_symbol(name)
    local key_symbols = {
      ["escape"] = "⎋",
      ["clear"] = "⌧",
      ["delete left"] = "⌫",
      ["delete right"] = "⌦",
      ["eject"] = "⏏",
      ["shift"] = "⇧",
      ["control"] = "⌃",
      ["option"] = "⌥",
      ["command"] = "⌘",
      ["caps lock"] = "⇪",
      ["return"] = "⏎",
    }
    
    if key_symbols[name] ~= nil then
      return "{\\symbolfont{"..key_symbols[name].."}}"
    else
      return name
    end
  end
\end{luacode*}

\definecolor{GutesGreen}{rgb}{0.58,0.73,0.63}

\defaultfontfeatures{%
  Scale = MatchLowercase,%
  Contextuals = Alternate}
\defaultfontfeatures[\rmfamily]{Scale = 1}

\setmainfont{Kaius Pro}
\newfontfamily{\tabularfont}{Kaius Pro}[%
  Numbers = Monospaced]
\newfontfamily{\chapterfont}{Kaius Pro Light}[%
  Numbers = Lining]
\newfontfamily{\chaptercounterfont}{Kaius Pro}
\newfontfamily{\sectionfont}{Kaius Pro Bold}[%
  Numbers = {Lining, Monospaced}]
\newfontfamily{\subsectionfont}{Kaius Pro Bold Italic}[%
  Numbers = Monospaced]
\newfontfamily{\paragraphfont}{Kaius Pro Regular Italic}
\newfontfamily{\footnotereffont}{Kaius Pro Light}[%
  VerticalPosition = Superior]
\newfontfamily{\footnotemarkfont}{Kaius Pro}[%
  Numbers = {Monospaced, Lining}]
\newfontfamily{\patharrowfont}{Kaius Pro Light}[%
  Letters = Uppercase]
\newfontfamily{\latexlogoafont}{Kaius Pro Medium}
\newfontfamily{\arfont}{Lyon Arabic Text Regular}[%
  Script = Arabic]
\newfontfamily{\cjfont}{Kaiti.ttc}[%
  Path = \luadirect{tex.print(font_root("Kaiti TC"))},%
  Script = CJK]
\newfontfamily{\kofont}{AGChoiJeongHoScreen.otf}[%
  Path = \luadirect{tex.print(extra_font_root())},%
  Script = CJK,%
  Language = Korean]
\newfontfamily{\symbolfont}{GutenTagSymbols-Regular.otf}
\newfontfamily{\applelogofont}{Avenir Next}
\setsansfont{SkolarSansLatn-Rg.otf}[%
  Path = \luadirect{tex.print(extra_font_root())}]
\setmonofont{Codelia}
\newfontfamily{\mediumitalicmonofont}{Codelia Medium Italic}

\newcommand{\textRTL}[1]{\bgroup\textdir TRT{#1}\egroup}
\newcommand*{\capskip}{\hspace{\fontcharht\font`T}}

%%  Layout Formatting
\AtBeginDocument{%
  \setlength{\parindent}{1.8em}%
  \setlength{\baselineskip}{15pt}}

%% Fonts
% Levels
\setkomafont{chapter}%
  {\normalfont\chapterfont\Huge}
\setkomafont{section}%
  {\normalfont\sectionfont}
\setkomafont{subsection}%
  {\normalfont\subsectionfont}
\setkomafont{paragraph}%
  {\normalfont\paragraphfont}
% Table of Contents
\setkomafont{chapterentry}%
  {\normalfont\bfseries}
\setkomafont{footnote}%
  {\addfontfeature{LetterSpace=1.5}}

\pagestyle{headings}

\RedeclareSectionCommands[%
  toclinefill = {},%
  tocraggedpagenumber,%
  tocentrynumberformat = \tabularfont,%
]{chapter,section,subsection}
\RedeclareSectionCommands[%
  beforeskip = \baselineskip,%
  afterindent = false,%
  tocentryformat = \bfseries,%
]{chapter}
\RedeclareSectionCommands[%
  afterskip = 0.5\baselineskip,%
]{section}
\RedeclareSectionCommands[%
  tocentryformat = \itshape,%
]{subsection}
\RedeclareSectionCommand[%
  beforeskip = \baselineskip,%
]{paragraph}

\titleformat{\chapter}[display]%
  {\vspace{-7.3em}\Huge\chapterfont}
  {\hfill\raisebox{-1.2em}[0pt][0pt]{%
      \textcolor{black!16}{%
        \fontsize{3em}{3em}%
        \selectfont%
        \chaptercounterfont%
        \thechapter}}%
  \hspace{-0.1em}}%
  {0pt}%
  {}
  [\vspace{-0.7em}]

% Footnotes
\deffootnotemark%
  {{\footnotereffont\thefootnotemark}}

\deffootnote{1em}{1em}%
  {\footnotemarkfont\thefootnotemark\enskip}

\setfootnoterule[0pt]{0pt}

% Lists
\renewcommand{\descriptionlabel}[1]{\hspace{\labelsep}{#1}}

% Inline Styles
\urlstyle{rm}

% Logos
\setLaTeXa{\latexlogoafont\fontsize{9pt}{9pt}\selectfont A}
\setlogokern{La}{-0.32em}
\setlogokern{aT}{-0.1em}
\setlogodrop{0.37ex}

%% Inline Commands
% Symbols
\newcommand*{\AppleLogo}{{\applelogofont\symbol{"F8FF}}}
\newcommand*{\InlineSeparator}{\hspace{0.3em}{\symbolfont•}\hspace{0.3em}}

% UI Text
\newcommand*{\Label}[1]{\emph{#1}}
\newcommand*{\PathElement}[1]{\Label{#1}}
\newcommand*{\PathSeparator}{~{\patharrowfont→} }
\newcommand*{\Path}[1]{\luadirect{macos_userinterface_path(\luastring{#1})}}

\newcommand*\PrefKeyDef[1]{\normalfont\texttt{#1}}
\newcommand*\PrefKeyRef[1]{\normalfont\texttt{#1}}

\makeatletter
\newcommand*{\PrefKey}{%
  \@ifstar%
    \PrefKeyDef%
    \PrefKeyRef}
\makeatother

\newcommand*{\ButtonSymbol}[1]{%
  \raisebox{-0.15em}{\includegraphics[height=1em]{Images/#1 Button.pdf}}}

\NewDocumentCommand{\KeyboardEquivalent}{o m}{%
  \IfNoValueTF{#1}{}{{\symbolfont\luadirect{tex.print(modifier_key_symbols(\luastring{#1}))}}\kern1pt}%
  {\addfontfeatures{Letters=Uppercase}\luadirect{tex.print(key_symbol(\luastring{#2}))}}}

% Syntax Highlighting
\newcommand*{\Keyword}[1]{\textit{#1}}
\newcommand*{\ControlFlow}[1]{\textcolor{red!80!blue!70!black}{\mediumitalicmonofont#1}}
\newcommand*{\Storage}[1]{\textcolor{blue!65!green!85!black}{#1}}
\newcommand*{\String}[1]{\textcolor{red!80!black}{#1}}
\newcommand*{\Variable}[1]{\textcolor{green!45!black}{#1}}
\newcommand*{\Token}[1]{\textcolor{orange!55!white!45!black}{#1}}
\newcommand*{\Comment}[1]{\textcolor{black!65}{#1}}

%% Block Commands
% Screenshots
\newcommand*{\ScreenshotFigure}[1]{%
  \vspace{0.7\baselineskip}%
  \noindent#1%
  \vspace{0.5\baselineskip}}

% {file}{is_crop}{scale}{frame}{tc}{bc}
\newcommand*{\ScreenshotGraphic}[6]{%
  \luadirect{tex.print(screenshot_graphic(\luastring{#1}, #2, #3, \luastring{#4}, #5, #6))}}

% [frame][scale]{file}
\NewDocumentCommand{\Screenshot}{s O{0, 0, 0, 0} O{2} m}{%
  \ScreenshotGraphic%
    {#4}%
    {\IfBooleanTF{#1}{true}{false}}%
    {#3}%
    {#2}%
    {\IfBooleanTF{#1}{0}{1.2}}%
    {\IfBooleanTF{#1}{0}{1.4}}}

% Code Listings
\lstset{%
  basicstyle = \fontsize{11}{14}\selectfont\ttfamily,%
  flexiblecolumns = true,%
  keepspaces = true,%
  tabsize = 2,%
  frame = none,%
  aboveskip = 3mm plus 2pt minus 2pt,%
  belowskip = 3mm plus 2pt minus 2pt}

\lstnewenvironment{RichListing}%
  {\lstset{escapeinside = {<@}{@>}}}%
  {}

\newcommand*{\SetOperationWrapFigure}[1]{%
  \begin{wrapfigure}[0]{r}[4pt]{58pt}
    \vspace{-1.4em}{\includegraphics[width=58pt]{Images/Set Operations/#1.pdf}}
  \end{wrapfigure}}

\begin{document}
  \begin{titlepage}
    \includepdf[pages=1]{Cover.pdf}
  \end{titlepage}
  
  \nonfrenchspacing
  \raggedbottom
  
  \thispagestyle{empty}

\begin{center}
  \begin{align*}
    \text{\textRTL{\arfont{إزالة العلامات}}} & \text{\InlineSeparator} \text{\textRTL{\arfont{إضافة علامات}}} \\
    \text{\addfontfeature{Script=Latin,Language=Chechen}Přidat Značky} & \text{\InlineSeparator} \text{\addfontfeature{Script=Latin,Language=Chechen}Odstranit Značky} \\
    \text{\addfontfeature{Script=Latin,Language=English}Add Tags} & \text{\InlineSeparator} \text{\addfontfeature{Script=Latin,Language=English}Remove Tags} \\
    \text{\addfontfeature{Script=Latin,Language=Spanish}Añadir Etiquetas} & \text{\InlineSeparator} \text{\addfontfeature{Script=Latin,Language=Spanish}Suprimir Etiquetas} \\
    \text{\addfontfeature{Script=Latin,Language=German}Tags hinzufügen} & \text{\InlineSeparator} \text{\addfontfeature{Script=Latin,Language=German}Tags entfernen} \\
    \text{\addfontfeature{Script=Latin,Language=French}Ajouter des Tags} & \text{\InlineSeparator} \text{\addfontfeature{Script=Latin,Language=French}Supprimer les Tags} \\
    \text{\addfontfeature{Script=Latin,Language=Italian}Aggiungi Tag} & \text{\InlineSeparator} \text{\addfontfeature{Script=Latin,Language=Italian}Rimuovi Tag} \\
    \text{\cjfont\addfontfeature{Language=Japanese}タグを追加} & \text{\InlineSeparator} \text{\cjfont\addfontfeature{Language=Japanese}タグを削除} \\
    \text{\kofont{태그 추가}} & \text{\InlineSeparator} \text{\kofont{태그 제거}} \\
    \text{\addfontfeature{Script=Latin,Language=Portuguese}Adicionar Etiquetas} & \text{\InlineSeparator} \text{\addfontfeature{Script=Latin,Language=Portuguese}Remover Etiquetas} \\
    \text{\addfontfeature{Script=Cyrillic,Language=Russian}Добавить Метки} & \text{\InlineSeparator} \text{\addfontfeature{Script=Cyrillic,Language=Russian}Удалить Метки} \\
    \text{\addfontfeature{Script=Latin,Language=Turkish}Etiket Ekle} & \text{\InlineSeparator} \text{\addfontfeature{Script=Latin,Language=Turkish}Etiketleri Kaldır} \\
    \text{\cjfont\addfontfeature{Language={Chinese Simplified}}添加标签} & \text{\InlineSeparator} \text{\cjfont\addfontfeature{Language={Chinese Simplified}}移除标签} \\
    \text{\cjfont\addfontfeature{Language={Chinese Traditional}}新增標籤} & \text{\InlineSeparator} \text{\cjfont\addfontfeature{Language={Chinese Traditional}}移除標籤} \\
    \hspace{12em} & \phantom{\text{\InlineSeparator}} \hspace{12em} % hack to center align separator symbol
  \end{align*}  
  \vspace{-5em}
  
  \vfill
  
  \includegraphics[width=6em]{Images/Icon.pdf} \\
  \textbf{Guten Tag} \\
  
  \vfill
  
  © 2021 Florian Pircher \\
  {Licensed under \href{https://creativecommons.org/licenses/by/4.0/}{Attribution 4.0 International ({\lsstyle\addfontfeature{Letters={UppercaseSmallCaps,SmallCaps}}CC BY 4.0})}}
  
  \bigskip
  
  \bgroup
  \fontsize{8.5pt}{12pt}\selectfont
  
  \emph{\emph{OpenType} is a registered trademark of Microsoft~Corporation.} \\
  \emph{\emph{Python} is a registered trademark of the Python~Software~Foundation.} \\
  \emph{\emph{GitHub} is a exclusive trademark registered in the United~States by GitHub,~Inc.} \\
  \emph{\emph{Mac} and \emph{macOS} are trademarks of Apple~Inc., registered in the U.S. and other countries and regions.} \\
  \egroup
  
  \bigskip
  
  Typeset on \today.
\end{center}

\input{Chapters/Table of Contents.tex}
\chapter{Installation}%
\label{installation}

If you are reading this handbook on the Mac on which you are using Glyphs, simply click the button below to install Guten~Tag:

\bigbreak\noindent%
\href{https://formkunft.com/glyphs/plugins/guten-tag/install}{%
  \includegraphics{Images/Install Guten Tag in Glyphs.pdf}}

\section{Install from the Plugin Manager}%
\label{installation:plugin-manager}

In Glyphs, open the Plugin Manager by selecting \Path{Window > Plugin Manager > Plugins}.
Search for \enquote{Guten Tag} and click \Label{Install} next to the plugin preview.
Relaunch Glyphs for the plugin to be loaded.

\bigbreak\noindent\Screenshot[window]{Images/Plugin Manager.png}%

\section{Installation Issues}%
\label{installation:issues}

If the installation fails, feel free to contact me by opening a new topic on the Glyphs~Forum\footnote{\url{https://forum.glyphsapp.com}}.
You can also open an issue on the GitHub repository\footnote{\url{https://github.com/florianpircher/GutenTag}}.

\chapter{Guten Tag}%
\label{cha:guten_tag}

\noindent Guten~Tag adds a \emph{Tags} palette to the top right corner of your Glyphs window.
There, you can edit the tags for the currently selected glyphs.

\bigbreak\noindent%
\Screenshot[window]{Images/Glyphs window with empty tags field.png}
\bigbreak

\section{Adding \& Removing Tags}%
\label{sec:adding_and_removing_tags}

\Screenshot*[1656, 1086, 112, 232][1.5]{Images/Type tag name.png}\hfill
\Screenshot*[1656, 1086, 112, 232][1.5]{Images/Tokenize tag name.png}\hfill
\Screenshot*[1656, 1086, 112, 232][1.5]{Images/Type tag name with suggestion.png}

\medbreak\noindent Click on the tags field and start typing the name of a tag.
Tags can include letters, numbers, spaces, and other punctuation marks.
Only the comma (\texttt{,}) is special.
Typing a comma or pressing Return will add the tag and display its name in a blue token.
An autocompletion menu pops up when typing a tag that is already used within the font.
Accept a suggestion by pressing Return.

\noindent%
\Screenshot*[1656, 1086, 112, 232][1.5]{Images/Tag name selected.png}\hfill
\Screenshot*[1656, 1086, 112, 232][1.5]{Images/Tokenize tag name.png}\hfill
\Screenshot*[1656, 1086, 112, 232][1.5]{Images/Glyphs window with empty tags field.png}\hfill

\medbreak\noindent You can edit tags like normal text: drag the mouse cursor to select tags or use keyboard commands like \KeyboardEquivalent[shift]{←} or \KeyboardEquivalent[shift]{→}.
Selected tags appear with a white font on a dark blue background.
Delete tags by pressing the Delete~\KeyboardEquivalent{delete left} key.

\section{Batch Editing Tags}%
\label{sec:batch_editing_tags}

Editing tags from the tags field is possible when you have only one glyph selected or all selected glyphs have the same tags.
If, however, the tags of the selected glyphs are different then Guten~Tag can no longer show a unified tags field.
Instead, \enquote{Multiple Values} is displayed in the tags field and an additional row of controls appears at the bottom of the Guten~Tag palette.

\bigbreak\noindent%
\Screenshot[window]{Images/Multiple Selection.png}

\bigbreak\noindent Typing into a multiple-values field will overwrite the tags for all selected glyphs.
If you want to change tags individually without modifying any of the other tags, use the three buttons below the tags field.

\subsection{Add Tags}%
\label{sub:add_tags}

Pressing the Plus \ButtonSymbol{Plus} button allows you to add tags to the currently selected glyphs.
All existing tags of the selected glyphs will be maintained.

\medbreak\noindent%
\Screenshot*[286, 550, 286, 500]{Images/Batch Add Tags.png}

\subsection{Remove Tags}%
\label{sub:remove_tags}

Removing tags works similarly.
Press the Minus \ButtonSymbol{Minus} button and enter the tags that you want to remove from all selected glyphs.

\medbreak\noindent%
\Screenshot*[286, 550, 286, 500]{Images/Batch Remove Tags.png}

\subsection{Rename Tags}%
\label{sub:rename_tags}

You can also batch rename tags by pressing \Label{Rename}.
Select the tag that you want to rename and enter its new name.
You can also rename a tag to the name of another tag\kern0.05em, merging the two.

\medbreak\noindent%
\Screenshot*[286, 550, 286, 500]{Images/Batch Rename Tags.png}

\chapter{Using Tags}%
\label{cha:using_tags}

In the previous chapter, you have learned how to add and remove tags to and from glyphs.
This chapter introduces workflows for which tags can be helpful.

\section{Navigate Between Related Glyphs}%
\label{sec:navigate_between_related_glyphs}

Each tag has a small disclosure chevron next to its name.
Click on the chevron to open a menu containing all glyphs with the tag.
You can also Control-click anywhere on the tag to access the menu.

\medbreak\noindent\Screenshot[0, 54, 347, 0]{Images/Glyphs Preview Menu.png}

% \medbreak\noindent The menu presents each glyph for the selected tag with its name, image, Unicode values, and colors.
\medbreak\noindent The menu presents all glyphs belonging to the tag with their names, images, Unicode values, and colors.
Select a glyph from the menu to open it in the edit view.
If you are already in the edit view, the currently edited glyph gets replaced by the selected glyph.

A checkmark next to a glyph indicates the current glyph.
When multiple glyphs are selected, each selected glyph is marked by a horizontal line instead.

Selecting \Label{Show All Glyphs} at the top of the menu opens all glyphs with the selected tag in a new edit view tab.

The size of the glyph images and the number of glyphs to display in the menu can be adjusted.
See section~\ref{sec:glyph_preview_menu} on page~\pageref{sec:glyph_preview_menu} for details.

Glyph navigation using tags can be useful for a variety of cases.
You can tag glyphs by their shape (for example \emph{round} or \emph{square}),
by their proportions (\emph{narrow}, \emph{tabular},~…),
by their placement (\emph{top}, \emph{bottom}, \emph{overlay},~…),
relation to other glyphs (\emph{caseable}/\emph{case}, \emph{ligating},~…), or
design features (\emph{tail}, \emph{leg}, \emph{crossbar},~…).
This way, when you make changes to one glyph, you can quickly jump to the other glyphs with the same tags and apply the change there, too.

\section{Tags for OpenType Feature Code}%
\label{sec:tags_for_opentype_feature_code}

You can use tags in your feature code to create glyph classes.
The two expressions \texttt{\$["some-tag" in tags]} and \texttt{\$[tags contains "some-tag"]} are equivalent:
They expand into a space-separated list of glyph names for all glyphs with the tag \emph{some-tag}.
See the Glyphs tutorial on tokens\footnote{\url{https://glyphsapp.com/learn/tokens}} for a general introduction to the \texttt{\$[…]} notation.
In case of tags, the token

\begin{RichListing}
<@\hspace{\parindent}@><@\Token{\$[}@><@\String{"narrow"}@> <@\Keyword{in}@> <@\Storage{tags}@><@\Token{]}@>
\end{RichListing}

\noindent expands to

\begin{RichListing}
<@\hspace{\parindent}@>dotaccentcomb dotaccentcomb.case dotbelowcomb
\end{RichListing}

\noindent if the glyphs \emph{dotaccentcomb}, \emph{dotaccentcomb.case}, \& \emph{dotbelowcomb} have the tag \emph{narrow}.

Tokens allow you to describe what kind of glyphs you want in your glyphs classes.
The \texttt{\$["…" in tags]} expression shown above collects all glyphs for a given tag.
When you add or remove tags using Guten~Tag\kern0.05em, you don’t need to update your feature code.
All glyphs classes defined using tokens are updated automatically on export by Glyphs.

A glyphs class can be defined from the sidebar of the feature code editor or inline with other feature code.
If you create glyph classes in the sidebar, place the \texttt{\$[…]} token directly into the code editor:

\bigbreak\noindent\Screenshot[111, 0, 111, 75]{Images/Token Feature Code.png}

\noindent Otherwise, wrap the token in square brackets and assign it to a \texttt{@Class}.

\begin{RichListing}
<@\Variable{@Narrow}@> = [<@\Token{\$[}@><@\String{"narrow"}@> <@\Keyword{in}@> <@\Storage{tags}@><@\Token{]}@>];
\end{RichListing}

\noindent You can combine tags with other predicates:

\begin{RichListing}
<@\Variable{@NarrowMarks}@> = [<@\Token{\$[}@><@\String{"narrow"}@> <@\Keyword{in}@> <@\Storage{tags}@> <@\Keyword{and}@> <@\Storage{category}@> == <@\String{"Mark"}@><@\Token{]}@>];
<@\Variable{@NarrowCTA}@>   = [<@\Token{\$[}@><@\String{"narrow"}@> <@\Keyword{in}@> <@\Storage{tags}@> <@\Keyword{and}@>
                  <@\Storage{name}@> <@\Keyword{in}@> <@\Storage{class}@>(<@\Variable{CombiningTopAccents}@>)<@\Token{]}@>];
\end{RichListing}

\noindent Glyphs tokens can be used to perform set arithmetic on tags.
All basic set operations are presented in the following examples for the two tags \emph{A} and \emph{B}.

\paragraph{Identity}%
\label{par:identity}

\SetOperationWrapFigure{Id}

All glyphs that have the tag \emph{A}.

\begin{RichListing}
<@\Token{\$[}@><@\String{"A"}@> <@\Keyword{in}@> <@\Storage{tags}@><@\Token{]}@>
\end{RichListing}

\paragraph{Complement}%
\label{par:complement}

\SetOperationWrapFigure{Complement}

All glyphs that do not have the tag \emph{A}.

\begin{RichListing}
<@\Token{\$[}@><@\Keyword{not}@> <@\String{"A"}@> <@\Keyword{in}@> <@\Storage{tags}@><@\Token{]}@>
\end{RichListing}

\paragraph{Union}%
\label{par:union}

\SetOperationWrapFigure{Union}

All glyphs that have the tag \emph{A} or the tag \emph{B} (or both).

\begin{RichListing}
<@\Token{\$[}@><@\String{"A"}@> <@\Keyword{in}@> <@\Storage{tags}@> <@\Keyword{or}@> <@\String{"B"}@> <@\Keyword{in}@> <@\Storage{tags}@><@\Token{]}@>
\end{RichListing}

\paragraph{Intersection}%
\label{par:intersection}

\SetOperationWrapFigure{Intersection}

All glyphs that have both the tag \emph{A} and the tag \emph{B}.

\begin{RichListing}
<@\Token{\$[}@><@\String{"A"}@> <@\Keyword{in}@> <@\Storage{tags}@> <@\Keyword{and}@> <@\String{"B"}@> <@\Keyword{in}@> <@\Storage{tags}@><@\Token{]}@>
\end{RichListing}

\paragraph{Difference}%
\label{par:difference}

\SetOperationWrapFigure{Difference}

All glyphs that have the tag \emph{A} but not the tag \emph{B}.

\begin{RichListing}
<@\Token{\$[}@><@\String{"A"}@> <@\Keyword{in}@> <@\Storage{tags}@> <@\Keyword{and}@> <@\Keyword{not}@> <@\String{"B"}@> <@\Keyword{in}@> <@\Storage{tags}@><@\Token{]}@>
\end{RichListing}

\paragraph{Symmetric difference}%
\label{par:symmetric_difference}

All glyphs that either have the tag \emph{A} or the tag \emph{B}, but not both.
The symmetric difference can be expressed in two ways.

\SetOperationWrapFigure{Symmetric Difference}

\begin{RichListing}
<@\Comment{\# union of differences:}@>
<@\Token{\$[}@>(    <@\String{"A"}@> <@\Keyword{in}@> <@\Storage{tags}@> <@\Keyword{and}@> <@\Keyword{not}@> <@\String{"B"}@> <@\Keyword{in}@> <@\Storage{tags}@>) <@\Keyword{or}@>
  (<@\Keyword{not}@> <@\String{"A"}@> <@\Keyword{in}@> <@\Storage{tags}@> <@\Keyword{and}@>     <@\String{"B"}@> <@\Keyword{in}@> <@\Storage{tags}@>)<@\Token{]}@>
<@\Comment{\# one and only one:}@>
<@\Token{\$[}@>    (<@\String{"A"}@> <@\Keyword{in}@> <@\Storage{tags}@> <@\Keyword{or}@>  <@\String{"B"}@> <@\Keyword{in}@> <@\Storage{tags}@>) <@\Keyword{and}@>
  <@\Keyword{not}@> (<@\String{"A"}@> <@\Keyword{in}@> <@\Storage{tags}@> <@\Keyword{and}@> <@\String{"B"}@> <@\Keyword{in}@> <@\Storage{tags}@>)<@\Token{]}@>
\end{RichListing}

All examples above can be combined with all other token predicates:
Whether a glyph has components (\texttt{hasComponents == true}),
how many tags it has (\texttt{countOfTags > 5}),
if the name contains a suffix (\texttt{name like "*.alt"}),
how high an anchor of the glyph is (\texttt{layer0.anchors.top.y > 580}),
and many more.
Again, read the Glyphs tutorial on tokens to see all possibilities.

\section{Scripting With Tags}%
\label{sec:scripting_with_tags}

Tags are also useful for scripting.
The Glyphs API exposes the \texttt{tags} property\footnote{\url{https://docu.glyphsapp.com/Core/Classes/GSGlyph.html\#//api/name/tags}} on glyphs.
This property is wrapped by the Python Scripting API starting from Glyphs 3.0.3.
In Python, you can iterate over the tags of a glyph like so:

\begin{RichListing}
<@\ControlFlow{for}@> <@\Variable{tag}@> <@\ControlFlow{in}@> <@\Variable{someGlyph}@>.tags:
    <@\Storage{print}@>(<@\Variable{tag}@>)
\end{RichListing}

\medbreak\noindent\Screenshot[window]{Images/Macro Window Iterate Tags.png}

\bigbreak\noindent Set tags using the \texttt{setTags:} (in Python \texttt{setTags\_}) method:

\begin{RichListing}
<@\Variable{someGlyph}@>.<@\Storage{setTags\_}@>([<@\String{"caseable"}@>, <@\String{"narrow"}@>, <@\String{"top"}@>])
\end{RichListing}

\noindent There is also a \texttt{countOfTags} property as well as a variety of accessor methods:

\begin{description}
  \item[\texttt{addTag:}]
    Adds a tag to a glyph.
  
  \item[\texttt{removeObjectFromTags:}]
    Removes a tag from a glyph.
  
  \item[\texttt{removeObjectFromTagsAtIndex:}]
    Removes a tag by its index.
  
  \item[\texttt{indexOfObjectInTags:}]
    Returns the index of the given tag.
  
  \item[\texttt{objectInTagsAtIndex:}]
    Returns the tag for the given index.
\end{description}

\noindent Use the \texttt{allTags} method on a font to get the tags for all glyphs of that font.

\begin{RichListing}
<@\ControlFlow{for}@> <@\Variable{tag}@> <@\ControlFlow{in}@> <@\Variable{someFont}@>.allTags():
    <@\Storage{print}@>(<@\Variable{tag}@>)
\end{RichListing}

\noindent Python method names use an underscore (\texttt{\_}) instead of a colon (\texttt{:}).
You can find the full documentation at \url{https://docu.glyphsapp.com/Core/index.html}

\chapter{Preferences}%
\label{cha:preferences}

Guten Tag offers a range of configurable preferences.
Set preferences with the Macro panel (\Path{Window > Macro Panel}).

\medbreak\noindent\Screenshot[window]{Images/Macro Window Preferences.png}

\section{Glyph Preview Size}%
\label{preference:glyph-preview-size}

The \texttt{GutenTagGlyphPreviewSize} (\emph{type:} double, \emph{default:} \texttt{36}) preference defines the width and height of a glyph preview image in display points. The value must be a positive number.

Run the following line in the Macro panel to set the size (or set the value to \texttt{None} to use the default size):

\begin{RichListing}
<@\Variable{Glyphs}@>.defaults[<@\String{"GutenTagGlyphPreviewSize"}@>] = <@\Number{56}@>
\end{RichListing}

\section{Glyph Preview Inset}%
\label{preference:glyph-preview-inset}

The \texttt{GutenTagGlyphPreviewInset} (\emph{type:} double, \emph{default:} \texttt{4}) preference controls the inset on all four edges from a glyph preview image in display points. The font size of the glyph preview is as follows:
\begin{align*}
  \text{GutenTagGlyphPreviewSize − {\addfontfeature{Numbers = Lining}2} × GutenTagGlyphPreviewInset}
\end{align*}
Negatives values crop into the image.

Run the following line in the Macro panel to set the inset (or set the value to \texttt{None} to use the default inset):

\begin{RichListing}
<@\Variable{Glyphs}@>.defaults[<@\String{"GutenTagGlyphPreviewInset"}@>] = <@\Number{6}@>
\end{RichListing}

\section{Maximum Glyph Preview Count}%
\label{preference:maximum-glyph-preview-count}

The \texttt{GutenTagMaximumGlyphPreviewCount} (\emph{type:} long, \emph{default:} \texttt{1000}) preference limits the number of glyph previews shown in the tag menu. This preference is useful since creating glyph previews can be impossibly slow if there are many glyphs for a tag. The default limit of 1000 should be fast enough in most cases. Remove the limit by setting the value to a non-positive number (for example, \texttt{-1}).

Run the following line in the Macro panel to set the count (or set the value to \texttt{None} to use the default count):

\begin{RichListing}
<@\Variable{Glyphs}@>.defaults[<@\String{"GutenTagMaximumGlyphPreviewCount"}@>] = <@\Number{200}@>
\end{RichListing}

\chapter{Remarks}%
\label{cha:remarks}

If Guten~Tag does not work as it should or is missing a helpful feature, do not hesitate to contact me. This includes incorrect translations, typos in this handbook, or any other related issue.

\setlength{\tabcolsep}{0pt}
\medbreak\begin{tabular}{rl}
  Email  & \capskip\href{mailto:florian@addpixel.net}{\texttt{florian@addpixel.net}}\\
  GitHub & \capskip\url{https://github.com/florianpircher/GutenTag} \\
\end{tabular}

\bigbreak\noindent This handbook was typeset in Würzburg\kern0.05em, Germany
with {Lua\LaTeX} in \emph{Kaius} by Lisa~Fischbach and \emph{Codelia} by Toshi~Omagari,
both of whom generously made adjustments to their fine work to accomodate the needs of this handbook.

\bigbreak\noindent Additional fonts in use are \emph{Lyon~Arabic} by Khajag~Apelian, Wael~Morcos, and Kai~Bernau;
\emph{Kaiti} by Zhang~Jiasheng\kern0.05em, Zhou~Huanbin, and Chen~Lütan;
and \emph{Choieongho} by Joachim~Müller-Lancé, Ku~Moa, and Choi~Jeongho.

\bigbreak\noindent Special thanks to Georg~Seifert for reviewing early versions of the plugin code,
providing insightful comments on the internals of the plugin, and
extending the Glyphs~API to make working with tags more convenient.

\bigbreak

\begin{center}
  \includegraphics[height=1em]{Images/Tag.pdf} \\
  {\addfontfeature{Letters={UppercaseSmallCaps,SmallCaps}}\lsstyle{Gute N\kern-0.05emacht}}
\end{center}


\end{document}
